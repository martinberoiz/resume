% LaTeX Curriculum Vitae Template
%
% Copyright (C) 2004-2009 Jason Blevins <jrblevin@sdf.lonestar.org>
% http://jblevins.org/projects/cv-template/
%
% You may use use this document as a template to create your own CV
% and you may redistribute the source code freely. No attribution is
% required in any resulting documents. I do ask that you please leave
% this notice and the above URL in the source code if you choose to
% redistribute this file.

\documentclass[letterpaper]{article}

\usepackage{hyperref}
\usepackage{geometry}

% Comment the following lines to use the default Computer Modern font
% instead of the Palatino font provided by the mathpazo package.
% Remove the 'osf' bit if you don't like the old style figures.
\usepackage[T1]{fontenc}
\usepackage[sc,osf]{mathpazo}

% Set your name here
\def\name{Martin Beroiz}

% Replace this with a link to your CV if you like, or set it empty
% (as in \def\footerlink{}) to remove the link in the footer:
\def\footerlink{}

% The following metadata will show up in the PDF properties
\hypersetup{
  colorlinks = true,
  urlcolor = black,
  pdfauthor = {\name},
  pdfkeywords = {economics, statistics, mathematics},
  pdftitle = {\name: Curriculum Vitae},
  pdfsubject = {Curriculum Vitae},
  pdfpagemode = UseNone
}

\geometry{
  body={6.5in, 8.5in},
  left=1.0in,
  top=1.25in
}

% Customize page headers
\pagestyle{myheadings}
\markright{\name}
\thispagestyle{empty}

% Custom section fonts
\usepackage{sectsty}
\sectionfont{\rmfamily\mdseries\Large}
\subsectionfont{\rmfamily\mdseries\itshape\large}

% Other possible font commands include:
% \ttfamily for teletype,
% \sffamily for sans serif,
% \bfseries for bold,
% \scshape for small caps,
% \normalsize, \large, \Large, \LARGE sizes.

% Don't indent paragraphs.
\setlength\parindent{0em}

% Make lists without bullets
\renewenvironment{itemize}{
  \begin{list}{}{
    \setlength{\leftmargin}{1.5em}
  }
}{
  \end{list}
}

\begin{document}

% Place name at left
{\huge \name}

% Alternatively, print name centered and bold:
%\centerline{\huge \bf \name}

\vspace{0.25in}

\begin{minipage}{0.45\linewidth}
  Center for Gravitational Wave Astronomy \\
  \href{http://www.utb.edu/}{University of Texas Rio Grande Valley} \\
  One West University Blvd, TX 78520
\end{minipage}
\begin{minipage}{0.45\linewidth}
  \begin{tabular}{ll}
    Phone: & (956) 203-1958 \\
%    Fax: &  (919) 962-5678 \\
    Email: & \href{mailto:martinberoiz@gmail.com}{\tt martinberoiz@gmail.com} \\
    Homepage: & \href{http://www.martinberoiz.org}{\tt http://www.martinberoiz.org} \\
    GitHub: & \href{https://github.com/martinberoiz}{\tt https://github.com/martinberoiz}
  \end{tabular}
\end{minipage}


%\section*{Personal}

%\begin{itemize}
%\item Born on March 11, 1981, in Argentina.
%\item Home Address: 205 W Madison St., Apt 12. Brownsville, TX (78520).
%\end{itemize}


\section*{Education}

\begin{itemize}
  \item B.S. Physics, National Cordoba University (Argentina), 2007.
  \item M.S. Physics, University of Texas at Brownsville, 2010.
  \item PhD in Physics, University of Texas at San Antonio, 2017
\end{itemize}

 
\section*{Technical Skills}

\begin{itemize}
\item {\bf Languages:} C, C++, Python, Objective-C, Bash Scripting, Fortran.
\item {\bf Web Development:} HTML, CSS, Javascript, Django, Apache, SQL databases.
\item {\bf Operating Systems:} Linux servers, systemd background processes. OS X (Cocoa API) XCode development, OpenGL.
\item {\bf Dev Tools:} SVN and Git version control. Continuous Integration testing with Travis-CI, Tox, Pytest. Sphinx Documentation.
\item {\bf Programs:} Mathematica, Matlab, Maple.
\end{itemize}


\section*{Algorithms developed}
\begin{itemize}
\item {\bf LVC-GCN:} (Private GitHub repository) An alert system to receive, process, and disseminate Gravitational Wave GCNs from the LVK Collaboration. The selected targets of observation are relayed to the TOROS broker website and alerted to members through email and Slack notifications (systemd, Python, SQL, Secret Keys Management).
\item \href{https://toros.utrgv.edu/broker}{{\bf Astronomy broker website:}} A Django website to coordinate the assignment of observational targets for a night of observation between different observatories on the TOROS Network. (HTML, CSS, Javascript, Django, SQL ORM, Authentication backends) \href{https://toros.utrgv.edu/broker}{https://toros.utrgv.edu/broker}
\item \href{https://github.com/toros-astro/astroalign}{{\bf Astroalign:}} A Python module for the alignment of two stellar images without World Coordinate System information. \href{https://github.com/toros-astro/astroalign}{https://github.com/toros-astro/astroalign}
\item \href{https://github.com/toros-astro/ois}{{\bf OIS (Optimal Image Subtraction):}} A Python module with a C extension to find transient events on stellar images. Uses classic Difference Image Analysis techniques. \href{https://github.com/toros-astro/ois}{https://github.com/toros-astro/ois}
\item {\bf Software pipeline} for reduction and analysis of astronomical images. Algorithms developed: Optimal Image Differencing and Machine Learning classifier. (Python, Django web app)
\item \href{https://github.com/quatrope/AutoMix}{{\bf Refactoring of AutoMix:}} A C library for Reversible Jump MCMC Sampler with automatic proposal-distribution-fit developed by David Hastie. \href{https://github.com/quatrope/AutoMix}{https://github.com/quatrope/AutoMix}
\item {\bf Bayesian analysis} to estimate the influence of calibration errors on LIGO gravitational wave data. (C++)
\item {\bf Spectral Method} modeling algorithm to numerically solve a 2D helically-reduced wave equation on different geometries with mixed boundary conditions (Fortran algorithms added to a pre existing repository of spectral method routines, SVN, Mathematica). 
\item \href{https://martinberoiz.org/2013/11/11/geodesic_calculator/}{{\bf Numerical integrator}} of time-like geodesics (light paths) around a rotating Kerr Black Hole. 3D representation of the geodesic using OpenGL in an interactive OS X GUI. (C, Objective-C, OpenGL) \href{https://martinberoiz.org/2013/11/11/geodesic_calculator/}{https://martinberoiz.org/2013/11/11/geodesic\_calculator}
\item {\bf Prototype algorithm} for a real-time radio pulse detection, searching for ``dispersion measure'' signature in transient astronomical signals. (C, PGPLOT, FFTW, OpenMP)
\end{itemize}


\section*{Work experience}
\begin{itemize}
\item Assistant Research Scientist for the Center for Gravitational Wave Astronomy, University of Texas Rio Grande Valley. Oct 2017-Present.
\item System Administrator for the Department of Physics and Astronomy, University of Texas at Brownsville. Fall 2015
\item System Administrator Assistant for the Department of Physics and Astronomy, University of Texas at Brownsville. 2013- Spring 2015
\end{itemize}

\section*{Publications}

\begin{itemize}
\item \href{https://www.sciencedirect.com/science/article/pii/S221313372030038X}{\bf Astroalign: A Python module for astronomical image registration}. (2020) M. Beroiz et al. \emph{Astronomy and Computing} Volume 32, July 2020, 100384.
\item \href{http://doi.org/10.1093/mnras/stz3634}{ \bf TOROS Optical follow-up of the Advanced LIGO-VIRGO O2 second observational campaign}. (2019) Rodolfo Artola et al. \emph{Monthly Notices of the Royal Astronomical Society (MNRAS)} January 3, 2020.
\item \href{https://www.sciencedirect.com/science/article/pii/S2213133718300982}{\bf Machine Learning on Difference Image Analysis: A comparison of methods for transient detection}. (2019) Bruno Sanchez et al. {\it Astronomy and Computing} Volume 28, July 2019, 100284.
\item \href{https://arxiv.org/abs/1710.05844}{\bf Observations of the first electromagnetic counterpart to a gravitational wave source by the TOROS collaboration} (2017) M.C. D�az et al. {\it The Astrophysical Journal Letters} Volume 848, Number 2.
\item \href{https://arxiv.org/abs/1701.05566}{\bf Corral framework: Trustworthy and fully functional data intensive parallel astronomical pipelines} (2017) Juan B. Cabral, Bruno Sanchez, Martin Beroiz, Mariano Dominguez, Marcelo Lares, Sebastian Gurovich, Pablo Granitto; {\it Astronomy and Computing}, Volume 20, 140-154.
\item \href{http://adsabs.harvard.edu/abs/2016ApJ...828L..16D}{\bf GW150914: First search for the electromagnetic counterpart of a gravitational-wave event by the TOROS collaboration} (2016). Mario C. Diaz et al; {\it The Astrophysical Journal Letters}, Volume 828, Issue 2, article id. L16, 6 pp.
\item \href{http://adsabs.harvard.edu/abs/2016ApJ...826L..13A}{\bf Localization and broadband follow-up of the gravitational-wave transient GW150914} (2016). B. P. Abbott et al. {\it The Astrophysical Journal Letters}, Volume 826, Issue 1, article id. L13, pp
\item \href{http://www.math.unm.edu/~lau/DMS1216866/tauFluids_BHLP.pdf}{\bf Multidomain, sparse, spectral-tau method for helically symmetric flow} (2014). M Beroiz, T Hagstrom, SR Lau, RH Price; {\it Computers and Fluids}  (2014), pp. 250-265.
\item \href{http://iopscience.iop.org/1538-3881/147/5/100}{\bf Bright microwave pulses from PSR B0531+21 observed with a prototype transient survey receiver} April 2014. J. Andrew O'Dea, F. A. Jenet, Tsan-Huei Cheng, Chau M. Buu, Martin Beroiz, Sami W. Asmar, and J. W. Armstrong; {\it Astronomical Journal}, 147, 100.
\item \href{http://ieeexplore.ieee.org/xpls/abs_all.jsp?arnumber=5555950}{\bf A Prototype Radio Transient Survey Instrument for Piggyback Deep Space Network Tracking}, May 2011. Chau M. Buu, Fredrick A. Jenet, John W. Armstrong, Sami W. Asmar, Martin Beroiz, Tsan-Huei Cheng, and J. Andrew O'Dea; {\it Proceedings of the IEEE} Volume 99, Number 5.
\item \href{http://journals.aps.org/prd/abstract/10.1103/PhysRevD.76.024012}{\bf Gravitational instability of static spherically symmetric Einstein-Gauss-Bonnet black holes in five and six dimensions}, 2007. M. Beroiz, G. Dotti y R.J. Gleiser, {\it Physical Review D} 76, 024012 hep-th/0703074.
\end{itemize}


\section*{Scientific Meetings and Conferences}

\begin{itemize}
\item \href{http://meetings.aps.org/Meeting/APR19/Session/Q16.7}{Speaker at the APS April Meeting 2019}, abstract Q16.00007. "Gravitational-Wave Optical Counterpart Detection Methods for the TOROS Campaign During LVC O2 Observation Run". Denver, Colorado. April 2019.
\item Building Astronomy in Texas Symposium 2019 (BAT2019). Dallas, Texas. January 26- 27, 2019.
\item \href{https://jsi.astro.umd.edu/conferences/2018-jsi-workshop/workshop-home-page}{Joint Space-Science Institute Gravitational Wave Physics and Astronomy Workshop 2018}. Poster presentation: ``TOROS optical follow-up during the LVC O2 observational campaign''. College Park, Maryland, December 1-4, 2018.
\item \href{http://adsabs.harvard.edu/abs/2016APS..APRR14005B}{Speaker at the APS April Meeting 2016}, abstract R14.005. "Results of optical follow-up observations of advanced LIGO triggers from O1 in the southern hemisphere". South Lake City, Utah. April 2016.
\item \href{https://conference.scipy.org/scipy2014/schedule/presentation/1730/}{SciPy Conference 2014. Speaker at Mini Symposium in Astronomy}. "Transient detection and image analysis pipeline for TOROS project". Austin, Texas. July 2014.
\item Advances and Challenges in Computational General Relativity, Brown University, Providence, RI. May 20-22, 2011.
\item Poster presentation 215th American Astronomical Society (AAS) Meeting, Washington DC, January 3-7, 2010 ``The X-ray Emission of SN1978K: Still Here After All These Years''  - Eric M. Schlegel, M. Beroiz.
\item Poster presentation 215th American Astronomical Society (AAS) Meeting, Washington DC, January 3-7, 2010 ``Current Results at PALFA Pulsar Survey at Arecibo Observatory'' - M. Beroiz, K. Stovall, F. Jenet, J. Cordes, D. Lorimer, D. Backer, PALFALFA Consortium.
\item Summer Provost Research Program at UTSA, 2009. Research on X-ray spectrum of Super Nova 1978K and Poster Presentation.
\item UTB Summer School in Gravitational Wave Astronomy at South Padre Island TX, June 2008.
\item Grav07, La Falda (Cordoba), Argentina. November 5-7, 2007.
\item Poster Presentation at 92nd AFA (Physics Association Argentina) Annual Meeting, Salta, Argentina, September 24-27, 2007.
\end{itemize}

\section*{Teaching Experience}

\begin{itemize}
\item Lab Assistant for the Dept. Physics UT Rio Grande Valley - Aug 2015 - Aug 2017.
\item Teacher Assistant  - Thermodynamics. University of Texas at Brownsville, Aug-Dec 2009.
\item Lab Assistant for Dept. Physics, University of Texas at Brownsville. Jan-May 20014.
\item Teacher Assistant for Electrodynamics, University of Texas at Brownsville. Aug-Dec 20014.
\end{itemize}

\bigskip

% Footer
\begin{center}
  \begin{footnotesize}
    Last updated: \today \\
    \href{\footerlink}{\texttt{\footerlink}}
  \end{footnotesize}
\end{center}

\end{document}

